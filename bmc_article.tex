%% BioMed_Central_Tex_Template_v1.06
%%                                      %
%  bmc_article.tex            ver: 1.06 %
%                                       %

%%IMPORTANT: do not delete the first line of this template
%%It must be present to enable the BMC Submission system to
%%recognise this template!!

%%%%%%%%%%%%%%%%%%%%%%%%%%%%%%%%%%%%%%%%%
%%                                     %%
%%  LaTeX template for BioMed Central  %%
%%     journal article submissions     %%
%%                                     %%
%%          <8 June 2012>              %%
%%                                     %%
%%                                     %%
%%%%%%%%%%%%%%%%%%%%%%%%%%%%%%%%%%%%%%%%%


%%%%%%%%%%%%%%%%%%%%%%%%%%%%%%%%%%%%%%%%%%%%%%%%%%%%%%%%%%%%%%%%%%%%%
%%                                                                 %%
%% For instructions on how to fill out this Tex template           %%
%% document please refer to Readme.html and the instructions for   %%
%% authors page on the biomed central website                      %%
%% http://www.biomedcentral.com/info/authors/                      %%
%%                                                                 %%
%% Please do not use \input{...} to include other tex files.       %%
%% Submit your LaTeX manuscript as one .tex document.              %%
%%                                                                 %%
%% All additional figures and files should be attached             %%
%% separately and not embedded in the \TeX\ document itself.       %%
%%                                                                 %%
%% BioMed Central currently use the MikTex distribution of         %%
%% TeX for Windows) of TeX and LaTeX.  This is available from      %%
%% http://www.miktex.org                                           %%
%%                                                                 %%
%%%%%%%%%%%%%%%%%%%%%%%%%%%%%%%%%%%%%%%%%%%%%%%%%%%%%%%%%%%%%%%%%%%%%

%%% additional documentclass options:
%  [doublespacing]
%  [linenumbers]   - put the line numbers on margins

%%% loading packages, author definitions

\documentclass[twocolumn]{bmcart}% uncomment this for twocolumn layout and comment line below
%\documentclass{bmcart}
\usepackage{blindtext}
%%% Load packages
\usepackage{amsthm,amsmath}
\RequirePackage{natbib}
\RequirePackage[authoryear]% uncomment this for author-year bibliography

\RequirePackage{hyperref}
\usepackage[utf8]{inputenc} %unicode support
%\usepackage[applemac]{inputenc} %applemac support if unicode package fails
%\usepackage[latin1]{inputenc} %UNIX support if unicode package fails


%%%%%%%%%%%%%%%%%%%%%%%%%%%%%%%%%%%%%%%%%%%%%%%%%
%%                                             %%
%%  If you wish to display your graphics for   %%
%%  your own use using includegraphic or       %%
%%  includegraphics, then comment out the      %%
%%  following two lines of code.               %%
%%  NB: These line *must* be included when     %%
%%  submitting to BMC.                         %%
%%  All figure files must be submitted as      %%
%%  separate graphics through the BMC          %%
%%  submission process, not included in the    %%
%%  submitted article.                         %%
%%                                             %%
%%%%%%%%%%%%%%%%%%%%%%%%%%%%%%%%%%%%%%%%%%%%%%%%%


\def\includegraphic{}
\def\includegraphics{}



%%% Put your definitions there:
\startlocaldefs
\endlocaldefs


%%% Begin ...
\begin{document}

%%% Start of article front matter
\begin{frontmatter}

\begin{fmbox}
\dochead{Research}

%%%%%%%%%%%%%%%%%%%%%%%%%%%%%%%%%%%%%%%%%%%%%%
%%                                          %%
%% Enter the title of your article here     %%
%%                                          %%
%%%%%%%%%%%%%%%%%%%%%%%%%%%%%%%%%%%%%%%%%%%%%%

\title{Towards Strain-Based Structural Health Monitoring of Civil Structures Based on Self-Sensing Concrete Nanocomposites}

%%%%%%%%%%%%%%%%%%%%%%%%%%%%%%%%%%%%%%%%%%%%%%
%%                                          %%
%% Enter the authors here                   %%
%%                                          %%
%% Specify information, if available,       %%
%% in the form:                             %%
%%   <key>={<id1>,<id2>}                    %%
%%   <key>=                                 %%
%% Comment or delete the keys which are     %%
%% not used. Repeat \author command as much %%
%% as required.                             %%
%%                                          %%
%%%%%%%%%%%%%%%%%%%%%%%%%%%%%%%%%%%%%%%%%%%%%%

\author[
   addressref={aff1},                   % id's of addresses, e.g. {aff1,aff2}
   corref={aff1},                       % id of corresponding address, if any
   %noteref={n1},                        % id's of article notes, if any
   email={diego.castanedas@upb.edu.co}   % email address
]{\inits{DL}\fnm{Diego L} \snm{Castañeda-Saldarriaga}}
\author[
   addressref={aff1},
   email={joham.alvarez@upb.edu.co}
]{\inits{JA}\fnm{Joham} \snm{Alvarez-Montoya}}
\author[
   addressref={aff2},
   email={hader.martinez@upb.edu.co}
]{\inits{VTM}\fnm{Vladimir} \snm{Martínez-Tejada}}
\author[
   addressref={aff1},
   email={julian.sierra@upb.edu.co}
]{\inits{JSP}\fnm{Julián} \snm{Sierra-Pérez}}

%%%%%%%%%%%%%%%%%%%%%%%%%%%%%%%%%%%%%%%%%%%%%%
%%                                          %%
%% Enter the authors' addresses here        %%
%%                                          %%
%% Repeat \address commands as much as      %%
%% required.                                %%
%%                                          %%
%%%%%%%%%%%%%%%%%%%%%%%%%%%%%%%%%%%%%%%%%%%%%%

\address[id=aff1]{%                           % unique id
  \orgname{Grupo de Investigación en Ingeniería Aeroespacial (GIIA), Escuela de Ingenierías; Universidad Pontificia Bolivariana}, % university, etc
  %\street{Waterloo Road},                     %
  \postcode{055030}                                % post or zip code
  \city{Medellín},                              % city
  \cny{Co}                                    % country
}
\address[id=aff2]{%
  \orgname{Grupo de Investigación en Nuevos Materiales (GINUMA); Escuela de Ingenierías; Universidad Pontificia Bolivariana},
    %\street{Waterloo Road},                     %
  \postcode{055030}                                % post or zip code
  \city{Medellín},                              % city
  \cny{Co}                                    
}

%%%%%%%%%%%%%%%%%%%%%%%%%%%%%%%%%%%%%%%%%%%%%%
%%                                          %%
%% Enter short notes here                   %%
%%                                          %%
%% Short notes will be after addresses      %%
%% on first page.                           %%
%%                                          %%
%%%%%%%%%%%%%%%%%%%%%%%%%%%%%%%%%%%%%%%%%%%%%%

\begin{artnotes}
%\note{Sample of title note}     % note to the article
%\note[id=n1]{Equal contributor} % note, connected to author
\end{artnotes}



%%%%%%%%%%%%%%%%%%%%%%%%%%%%%%%%%%%%%%%%%%%%%%
%%                                          %%
%% The Abstract begins here                 %%
%%                                          %%
%% Please refer to the Instructions for     %%
%% authors on http://www.biomedcentral.com  %%
%% and include the section headings         %%
%% accordingly for your article type.       %%
%%                                          %%
%%%%%%%%%%%%%%%%%%%%%%%%%%%%%%%%%%%%%%%%%%%%%%

\begin{abstractbox}

\begin{abstract} % abstract 
Self-sensing materials are emerging as a promising technological development for the construction industry, where novel materials with the capability to provide information about the structural integrity and operating as structural material are required. From this perspective, several researchers had including different types of filler (carbon nanotubes, conductive fillers, polymer fillers, etc) inside cement-based matrix in order to obtain self-sensing and self-healing concrete. This article reports the manufacturing and multipurpose experimental characterization of a cement-based matrix (CBM) with carbon nanotube (CNT) inclusions, in order to find potential application on the structural health monitoring (SHM) field for the damage detection stage. Methodologies based on current-voltage (I-V) curves, direct current (DC) and biphasic direct current (BDC) were performed to understand the electric resistance of the CNT/CBM. Self-sensing  behavior (variation of electrical resistance with applied strain) of the manufactured samples  were studies  using a compression test while electric resistance measure were taken. To evaluate the fault detection capability, CNT/CBM was embedded into a reinforced concrete bean (RC-bean) and tested by a three-point bending. Test results showed that all the manufactured samples exhibit an Ohmic response. The CNT/CBM self-sensing behavior exhibited a linear proportionality between the values of electrical resistance against the magnitude of the applied loads. Change in global stiffness associated with the damage occurrence on the beam from a variation in the electrical resistance measured in the material was successfully self-sensing using the manufactured composited.
\end{abstract}

%%%%%%%%%%%%%%%%%%%%%%%%%%%%%%%%%%%%%%%%%%%%%%
%%                                          %%
%% The keywords begin here                  %%
%%                                          %%
%% Put each keyword in separate \kwd{}.     %%
%%                                          %%
%%%%%%%%%%%%%%%%%%%%%%%%%%%%%%%%%%%%%%%%%%%%%%

\begin{keyword}
\kwd{Smart materials}
\kwd{structural health monitoring}
\kwd{self-sensing composites}
\kwd{carbon nanotubes}
\kwd{cementitious composites}
\kwd{nanocomposites}
\kwd{civil structures}
\end{keyword}

% MSC classifications codes, if any
%\begin{keyword}[class=AMS]
%\kwd[Primary ]{}
%\kwd{}
%\kwd[; secondary ]{}
%\end{keyword}

\end{abstractbox}

\end{fmbox}% uncomment this for twcolumn layout
%\end{fmbox}% comment this for two column layout
\end{frontmatter}

%%%%%%%%%%%%%%%%%%%%%%%%%%%%%%%%%%%%%%%%%%%%%%
%%                                          %%
%% The Main Body begins here                %%
%%                                          %%
%% Please refer to the instructions for     %%
%% authors on:                              %%
%% http://www.biomedcentral.com/info/authors%%
%% and include the section headings         %%
%% accordingly for your article type.       %%
%%                                          %%
%% See the Results and Discussion section   %%
%% for details on how to create sub-sections%%
%%                                          %%
%% use \cite{...} to cite references        %%
%%  \cite{koon} and                         %%
%%  \cite{oreg,khar,zvai,xjon,schn,pond}    %%
%%  \nocite{smith,marg,hunn,advi,koha,mouse}%%
%%                                          %%
%%%%%%%%%%%%%%%%%%%%%%%%%%%%%%%%%%%%%%%%%%%%%%

%%%%%%%%%%%%%%%%%%%%%%%%% start of article main body
% <put your article body there>

%%%%%%%%%%%%%%%%
%% Background %%
%%

\section{Introduction} \label{introduction}
It is a fact that materials and structures degrade with time and usage. Within the context of civil structures (e.g. buildings, bridges, tunnels, dams, among others), natural and man-made hazards (e.g. earthquakes, typhoons, hurricanes, fire or collisions) represent also threats to the structural integrity that may cause catastrophic failures involving economic and human losses \cite{Xu2017a}. This requires the implementation of inspection procedures in order to guarantee serviceability and reliability of structures in their long lifetimes. These inspections are often carried out at a fixed time intervals using different methods comprised within the field of nondestructive testing (NDT), where the quality or integrity of a component is determined nondestructively by interrogating one or several physical variables that are damage-sensitive \cite{Shull2002}.

Some NDT methods include visual testing (both direct and remote) \cite{Agnisarman2019}, ultrasonics \cite{Zhao2018}, acoustic emissions \cite{Meo2014}, infrared thermography \cite{Yamazaki2018}, ground penetrating radar techniques and electrical resistivity tomography \cite{Salin2018}. These methods commonly require human intervention that increases uncertainty and implies costly equipment and personnel. SHM overcomes these factors using approaches where sensors are permanently installed into the structure so that NDT is performed continuously or online \cite{Xu2017a}. In these regard, SHM can reduce costs by providing valuable information for maintenance management, improve reliability by the possibility of finding damages at incipient stages and improve future designs by making available value information about the performance of the current ones \cite{Ogai2018a}.

Recent advances in SHM for civil structures include the use of piezoelectric sensors \cite{Liao2019},  fiber optic sensors (FOS) \cite{Glisic2013, Barrias2019, Xu2019}, electrochemical sensors (e.g. potentiometric, amperometric and conductometric) \cite{Hu2011, Qiao2012} and self-sensing composite materials \cite{Tian2019a}. A deep review of the sensing technologies applied to civil structures was recently reported by Taheri \cite{Taheri2019a}. Considering the available sensing technologies, one concern when developing effective SHM systems is the selection of the type of sensor to be used. The resistance and durability of the sensor operating in harsh environments and its integrability to large structures are paramount. That is why self-sensing composites, materials that not only bear loads but provide measures as a response to an external stimulus, have raised the interest of the research community \cite{DAlessandro2016, Han2015a, Rana2016a, Yang2020a}.

As cement-based materials or cementitious composites (e.g. paste, mortar and concrete) are the most popular building materials \cite{Xu2017a}, much of the scientific research around self-sensing composites is related to such materials. Self-sensing cementitious composites, also known as smart concretes, are fabricated by mixing functional fillers (e.g. carbon fibers \cite{Baeza2013a,Teomete2015, Sarwary2019a}, carbon nanotubes \cite{Elkashef2015a}, steel fibers \cite{Kang2018b}, \cite{Ding2019}, graphite powder \cite{Simonova2018}, nickel powder \cite{Wang2015} , among others \cite{Tian2019a}). Particularly, CNTs have raised the interest in the last years, due to their high specific mechanical and transport properties \cite{Schumacher2014,Ubertini2016}. 

The incorporation of CNTs in cementitious materials can endow these materials with a piezoresistive behavior. The CNTs are materials that when they are subjected to strain, their electrical properties change up to two orders of magnitude, exhibiting a proportional and reversible piezoresistive response to the external stimulus \cite{Garcia-Macias2017, Garcia-Macias2017a, Minot2003}. This electromechanical behavior is explained by various authors as a change in the conductivity of the CNT due to the change in the energy band induced by the strain applied on the volume of this \cite{Minot2003, PHAM2008, Han2015, Njuguna2012, Tjong2009, XinxinSun2009}. 

On the other hand, as the matrix is dielectric, electrical properties are filler-dominated by the CNT networks and the composite’s electrical conductivity changes according to the strain conditions \cite{Tian2019a}. In other words, by adding conductive fillers in dielectric matrices such as CNTs, there is a change in conductivity which is proportional to the application of load (or what is the same, proportional to strain). In this way, concretes with CNTs as fillers can be used as strain sensors. Moreover, such composites can also be used for damage detection since CNT networks can also be disturbed by damages such as cracks.

Despite this approach is promising for SHM of civil structures and different studies have been conducted in the field including electromechanical characterization \cite{Ubertini2014, Meoni2018a, Liu2019, Kim2019a}, fabrication procedures \cite{DAlessandro2016, Parvaneh2019a} and modeling techniques \cite{Garcia-Macias2017, Eftekhari2014, Garcia-Macias2018b}, it is still considered in their infancy stage and a broad practical implementation is still not possible at the moment \cite{Taheri2019a}. 

There are some issues that need to be addressed before a wide implementation of this technology in infrastructures, such as quality and variability in fabrication, uniform dispersion of fillers in the matrix, measurement procedures allowing reliable and repeatable monitoring of conductivity and approaches to damage detection in full-scale civil engineering structures \cite{Tian2019a, DAlessandro2016a, Shi2019b}.

Compared with characterization and fabrication, little research has been conducted to investigate the integrability and damage monitoring capabilities of self-sensing cementitious composites in structural members made of reinforced concrete under complex load scenarios as expected in real-world modern infrastructures \cite{Lagason2016, You2017, Al-Dahawi2017a, Naeem2017a}. Damage monitoring in beams or columns greatly depends on the self-sensing composite configuration. 

Most of the reported studies have proposed the construction in bulk, namely, the whole member is made of the self-sensing composite providing higher sensitivity to damages \cite{Hannan2018a, Lagason2016, Al-Dahawi2017a, Downey2017a, Gupta2017a, You2017, Yldrm2018}. For example, \citet{Al-Dahawi2017a} fabricated engineered cementitious composites with different carbon-based materials including CNTs. Then, the beam specimens were loaded under four-point bending tests in the elastic and plastic region to sense damage. The results were suitable in the plastic region, however, upon unloading in the elastic region the authors found not successful results. 

\citet{Downey2017} developed a resistor mesh model to detect, localize and quantify damage in structures from self-sensing cementitious composites based on the hypothesis that the electrical resistance of any self-sensing conductive material depends on its strain and damage states. \citet{Yldrm2018} investigated the self-sensing capabilities of reinforced large-scale beams designed to fail under shearing in four-point bending tests. The authors successfully proved that increments in maximum shear crack width produce changes in the electrical resistance.

Nevertheless, when dealing with a practical implementation of such bulk fabrication, which often requires demanding laboratory procedures, does not seem cost-effective towards a broad implementation. Other alternatives include coatings \cite{Downey2017, Downey2018e, Nespor2018a}, in which one surface of the member is covered with a layer of the self-sensing composite, embedded \cite{Meoni2018a, Saafi2009, DAlessandro2017}, where the self-sensing composite is prefabricated into standard small-size sensors and then integrated to the member, and novel forms dealing with coatings aggregates \cite{Han2015b, Gupta2017a, Loh2015a}.

The embedded form has the advantage of being less expensive and easier to integrate for practical applications. Civil structures can be designed with a network of such sensors to allow distributed sensing over large areas without needing a high amount of material and disturbing the structural performance. Additionally, this form is suitable for SHM of existing infrastructure as opposed to the bulk form which may be restrained to new constructions. 

\citet{Saafi2009} proposed CNT/cement sensors embedded into concrete beams for crack detection under three-point bending tests. The authors demonstrated that sudden changes in the electrical resistance are indicative of crack initiation. \citet{DAlessandro2016b} embedded cement-based sensors doped with CNTs for strain monitoring of reinforced concrete beams. The authors proposed a damage detection strategy based on vibration tests that were carried out by exciting the beam with a hammer. Spectral analysis of the results obtained with the smart sensors and with strain gauges included for validation demonstrated similar results, therefore, vibration-based SHM is feasible with these sensors. \citet{Lim2017} performed crack onset monitoring by embedding  cementitious composite sensors in reinforced mortar beams, the authors based their damage detection strategy on that if the cracks nucleate in the concrete, it could be propagated to the sensor and, therefore, the conductivity will change. \citet{Naeem2017a} evaluated the feasibility of crack sensing using CNT/cement composites in reinforced mortar beams. The authors found that steep changes in the resistance occurred at failure of the mortar under flexural loading. 

The damage detection approaches reviewed above rely on damages or cracks propagating to the cementitious composite sensors to change the CNT network and, hence, change resistance measurements. However, in real scenarios non-allowable damages cannot penetrate the sensor leading to misclassifications. Based on previous work related to strain-based SHM \cite{Sierra-Perez2015, Sierra-Perez2018}, it is proposed to use resistance measurements (correlated with strain measurements) to detect damages relying on that a damage occurrence (e.g. cracks, inclusions, corrosion, among others) affects the local stiffness of the structure and, therefore, the strain field is modified.  

The aim of this paper is to present a multi-purpose study that includes the characterization of cementitious composites with inclusions of CNTs, different test procedures and a proof-of-concept demonstration in a simply supported beam made of reinforced concrete of the suitability of these types of sensors for strain-based SHM. The novelty of this study focuses on the effective integration of the self-sensing concrete sensors in a structural member and in using information from them for damage detection based on strain demonstrating their suitability for future practical SHM applications.

This paper is divided into four sections, after this brief introduction, the experimental procedures to develop the specimens under study and their characterization methods including I-V obtaining curves, DC-based electrical characterization and characterization of piezoresistive behavior. After that, the results of such characterizations are shown in conjunction with the proof-of-concept in the structural member for SHM purposes.


\section{Experimental procedure.} \label{Experimental_procedure}

The experimental procedure used in this work was divided into two parts: the first one is focused on the dispersion of CNTs into a cement-based matrix (CBM), having as a main objective to obtain an uniform distribution of CNTs within the matrix. The second part is focused on a electromechanical behavior  and topographic characterization of the CBM/CNT, using a scanning electron microscopy (SEM) and DC/BDC methodologies. 

\subsection{CNT dispersion and samples manufacturing.} \label{materials}

For the CBM, Portland cement, quartz sand and water were used, in a water-to-cement ratio of 1:3 following ASTM C109 standard \cite{ASTMC1092000}. The CNTs used in this work were multi-walled type, from 10 to 20 micrometers in length and 50 to 80 nanometers in diameter supplied by Nanostructured and Amorphous Materials Inc company.

As dispersing agent, sodium dodecyl sulfate (SDS) was used in order to avoid the agglomeration of CNTs within the CBM, to improve the dispersion quality and to increase the concrete workability. In previous works \cite{Castaneda-Saldarriaga2019, Kyrylyuk2008, Shao2017, Myung2014, Sasmal2017, Rehman2018}, dispersants such as Triton X-100, Tween 20 and SDS were evaluated in order to determine the proportion and mixtures that improved the CNT dispersion. These studies concluded that the SDS is the most suitable in terms of avoiding the agglomeration of CNTs.

In order to estimate what percentage above the percolation threshold (defined as the minimum fraction of CNTs necessary for the material to exhibit electrical conductivity) generates better piezoresistive properties, CNT percentages by weight ($\%wt$) of 0.2, 0.5 and 0.8 of the total weight of the CBM were selected following the experimental methodologies proposed in the literature \cite{Coppola2011, Downey2017a, Cui2013, Garcia-Macias2017, Baeza2013a, Yoo2018a}. 

Since piezoresistive effect considerably depends on a proper CNT dispersion and these tend to agglomeration due to Van der Waals forces, the used methodology started by preparing a solution of SDS dispersant with a water-to-dispersant ratio of 10:1 \cite{Shao2017, Noh2013, Lee2017, Collins2012}.To obtain the water/dispersant solution (see Figure \ref{fig1}a), a BRANSON S-450D ultrasonic sonicator  with a maximum power of 400 Watts and tip of 6.35 mm was used for five minutes (1 second on and two seconds off, so that overheating the sample was avoided) \cite{Konsta-gdoutos2010, Konsta-Gdoutos2014}. Once a solution between water and dispersing agent was obtained, the different proportions of CNTs were added to the solution and the ultrasonic sonicator was used for another 45 minutes in the same configuration (Figure \ref{fig1}b). The ultrasonic sonicator power was set at 12 W for both processes.

\begin{figure}[h!]
  \caption{\csentence{Manufacturing process of the CBM/CNT composite.}
      }
   \label{fig1}
      \end{figure}

Simultaneously, according to the ASTM C109 standard \cite{Qiao2012}, the quantities of sand and cement for the manufacture of concrete samples (as shown in Figure \ref{fig1}c) were mixed with a mechanical stirrer. The next step was to pour the water/dispersant/CNTs solution to the cement/sand mixture.  Thereupon, the mixture was homogenized using a mechanical stirrer at a constant speed of 200 rpm (Figure \ref{fig1}d) following ASTM C1329 and ASTM C109 standards \cite{Collins2012, AmericanSocietyforTestingandMaterials2014}.

\begin{figure}[h!]
  \caption{\csentence{CBM/CNT specimens and copper electrodes mesh type configuration.}
  \label{fig2}
      }
      \end{figure}

Finally, the  CBM/CNT composite was cast in cube molds of 52 $\times$ 52 $\times$ 52 mm, following ASTM C109 \cite{ASTMC1092000} (Figure \ref{fig1}e), whereby copper electrodes were arranged in mesh-like configuration (as shown in Figure \ref{fig2}), so that it were embedded in the material. Copper was selected as the electrode material due to its low electrical resistance, which guarantees that the electrical resistance measurements of the composite material obtained mostly belong to the material and not to the electrodes \cite{Kim2016}. All CBM/CNT samples were naturally aged for 28 days under ACI 308 standard \cite{ACICommittee3082016}.




\subsection{Electrical behavior and topographic characterization of the CBM/CNT material.}


The topographic images of the CBM/CNT were obtained in order to analyzed the CNT dispersion within the CBM, for this purpose a SEM was used. To characterize the Ohmic behavior of the CBM/CNT composite, a I-V curve test  was implemented following the work performed by \citet{Han2015a}. This characterization helps to select the measuring methods and the measuring equipment. A comparison was also carry out between the use of techniques that involve direct current (DC) in order to find an appropriated methodology to measure the electrical properties of the CBM/CNT materials. Finally, the methodology used to characterized the self-sensig phenomenon in CBM/CNT and RC-beam is presented.

\subsubsection{Scanning electron microscopy of CBM/CNT material}\label{SEM}
After aging, specimens with different concentrations of CNTs were characterized using SEM. A Jeol JCM-6000 SEM was used in order to find a relationship among the concentration of CNTs, its dispersion and the electrical resistance of each specimen. A thin gold/palladium coating was deposited on the surface of the samples so as to avoid overload phenomena in dielectric samples when they are irradiated with the electron beam.

\subsubsection{Ohmic characterization of the CBM/CNT material.}\label{Ohm}

Characterizing the Ohmic behavior of the material involves measuring its  amperometric response when a voltage ($V$) input is applied in order to obtain the I-V curves. For this purpose, a typical assembly  of an ammeter, a DC power supply and a 10 kOhm resistor was used for obtaining I-V curves of each specimen \cite{YORKE1981200}.


I-V curves were obtained by sequential voltage increase applied between the electrodes and measuring the current change through the CBM/CNT. Measuring the Current through the CBM/CNT required an electrical resistor in series with one of the CBM/CNT electrodes as is shown in Figure \ref{fig3}a. Afterwards, an UNIT-T digital multimeter model UT39A was used to measure the current flowing through the specimen. The applied electrical voltage ranges from 0 to 9 volts with 0.5 V steps. Three sequential repetitions were performed for each sample.


\subsubsection{Electrical resistance characterization of the CBM/CNT using DC and BDC procedures.}\label{DC_BDC}

Once the Ohmic behavior was validated  means of the I-V curves analysis, DC and BDC methodologies were implemented to characterize the  CBM/CNT electrical resistance. The experimental setup for the electrical resistance measurement using DC, required just a digital multimeter (Fluke model 112), as shown in Figure \ref{fig3}b. Sampling rate frequency was set in 1 Hz \cite{Downey2017, Coppola2011, Dong2016}.

On the other hand, BDC was selected in order to avoid the polarization effect generated by the modification of the charge distribution that occurs in dielectric materials when a electric field is applied. BDC uses a square DC wave that is responsible for discharging the molecules applying an electronic flow in opposite direction to the power supply voltage \cite{Downey2017a, BOTTCHER1973}.

The methodology for CBM/CNT electrical resistance measuring  using BDC was proposed by  \citet{Downey2017a}. According to the authors, a four-probe method type assembly is recommended. For this purpose, a two-phase DC source with a $50\%$ duty cycle square wave output signal was used.  To perform the electrical characterization using BDC a 5 V peak-peak voltage between the specimen electrodes $V_EXT$ was applied with 1 Hz wave frequency. To measure the voltages ($V_EXT$ and $V_INT$), two Tektronix brand oscilloscopes (model TDS 1012C - EDU) were used, with a resolution of 0.02 V at a data sampling rate of 1 Hz.  Only the positive values of the square signal supplied to the circuit were take. A $R_0$ resistor of 1 kOhm was connected in serie to the circuit in order to determine the current flowing through the CBM/CNT, as is shown in Figure \ref{fig3}c.

Finally, to determine the electrical resistance of the composite material, the average value of $V_EXT$ set in the $80\%$ of the positive wave was taken and multiplied by $R_0$ using the Ohm's law, finding in this way, the current ($I$) flowing through the CBM/CNT. Then, the average value of $V_INT$ set in $80\%$ of the positive wave was taken and divided by the current ($I$), finding the resistance of the CBM/CNT.

\begin{figure}[h!]
  \caption{\csentence{a) Electrical connection diagram for ohmic characterization of the CBM/CNT. b) Electrical connection diagram for characterization of the electrical resistance using DC source. c) Electrical connection diagram for characterization of the electrical resistance using BDC.}
      }
  \label{fig3}
      \end{figure}


\subsubsection{CBM/CNT piezoresistive behavior characterization.}\label{Piezo}

To evaluate the electrical and mechanical behavior of the CBM/CNT material, the electrical resistance was measured while an uniaxial compressive test was carried out using an INSTRON 5582 universal testing machine (see Figuure \ref{fig3}b). Each specimen was instrumented with a Fiber Bragg Grating (FBG) bonded on the surface of one of the faces of the CBM/CNT subjected to compression with the aim of measuring strain. A fiber optic interrogator (Micron Optics SM130-700) was used for acquiring the wavelength shift as function on a mechanical stimulus applied to the CBM/CNT. The electrical resistance and strain values were acquired at a frequency rate of 1 Hz.

The maximum magnitude of the applied load applied was 5 kN. No more than 120 $\mu\epsilon$ were obtained for  any measurement. This strain level ensures no damages in the CBM/CNT specimens taking into account the Young modulus ans strength of the Portland cement used. Figure \ref{fig4} shows the experimental setup used for the piezoresistive characterization of the material. 

\begin{figure}[h!]
  \caption{\csentence{Experimental setup for CBM/CNT piezoresistivity determination.}
      }
    \label{fig4}
      \end{figure}

To correlate the variation of the electrical resistance and the strain, the expression described in Equation \ref{eq1} was used:

\begin{eqnarray}\label{eqexpmuts}
\lambda = \frac{\delta R}{\sigma R},
\label{eq1}
\end{eqnarray}

    
where $R$ is the initial resistance, $\delta R$ is the difference between initial resistance and final resistance and $\sigma$ are the strain measured by the FBG \cite{Pisello2017, Jang2017}.

\subsubsection{Health monitoring of a reinforced concrete beam using the CBM/CNT material.}\label{SHM_beam}

Once the CBM/CNT piezoresistive behavior  was characterized, a CBM/CNT was embedded into a RC-beam in order to asses its self-sensig response. For this purpose, a CBM/CNT coupon was made by using the same experimental procedure related in section \ref{DC_BDC} but having a different geometry (i.e. parallelepiped instead of cube). 

To ensure the load transfer between the RC-beam and the CBM/CNT, corrugated steel rods (dowels) of 6.35 mm diameter and 80 mm length were used (Figure \ref{fig5}a). Half of its length  was embedded into the CBM/CNT  parallelepiped and the remaining into the RC-beam, as can be seen in Figure \ref{fig5}c. The central section of the CBM/CNT coupon remains without corrugated steel rods as seeing in Figure \ref{fig5}a. 

In this way, besides the chemical bond which serves as interface between the concrete and corrugated steel, a mechanical bonding occurs between both of them. Therefore, the load transfer between the concrete beam and the CBM/CNT occurs through both effects.

To guarantee that the electrical properties of the CBM/CBM parallelepiped with corrugated steel rods did not differ from those exhibited by the specimens made by following ASTM C109 standard \cite{ASTMC1092000} described in section \ref{materials}., the following steps were proceeded:

\begin{itemize}
\item The same proportion of CNT and dispersion method were used, as well as the same proportions of cement, water, sand and dispersant.

\item The distance between the electrodes in the middle volume of the parallelepiped was ensured to be the same as that used in the specimens described in \ref{materials}.

\item The dimensions of the cross-section of the parallelepiped remained the same as those of the specimens described in section \ref{materials} and the measurements of the copper electrodes in mesh type configuration were the same as shown in Figure \ref{fig2}. 
\end{itemize}

The RC-beam (where the CBM/CBM parallelepiped was embedded) was made of same structural materials and proportions used to make the CBM as described in the section \ref{materials} (see Figure \ref{fig5}b) but without the addition of CNTs. Its final dimensions were 650 mm length, with a cross section of 160 $\times$ 130 mm.

The integration of the CBM/CNT parallelepiped within the RC-beam was carried out by positioning the specimen in the steel frame shown in Figure \ref{fig5}b. Hereinafter, the steel frame and CBM/CNT parallelepiped were placed in a wood formwork where the concrete was cast. Finally, the RC-beam was aged for 28 days. 

\begin{figure}[h!]
  \caption{\csentence{RC-beam manufacturing for the validation of the operation of the sensor against the health monitoring. a) CBM/CNT parallelepiped with corrugated steel rods. b) inclusion of the CNC/CBM parallelepiped within the steel frame of the RC-beam. c) Casting over the CNC/CBM parallelepiped and the reinforced steel structure of the concrete for the fabrication of the RC-beam.}
      }
      \label{fig5}
      \end{figure}


For the validation of the CBM/CNT parallelepiped embedded within the RC-beam as a strain sensor, a three-point bending test was performed. The distance between supports was set in 600 mm and a dynamic load was applied to the RC-beam from 0 to 14 N at a constant displacement speed of 5 mm/min. Measurements of the electrical resistance of the CBM/CNT were taken while applying load using the same methodology described in previous section. The experimental setup can be seen in Figure \ref{fig6}.

The main idea behind SHM based on strain measurements is to study the slope changes in a load vs. strain (which represents the global stiffness of the RC-beam) promoted by damage occurrence. A real damage (e.g. a crack) reduces the global stiffness and therefore, the slope of the curve decreases. On the other hand, a positive damage increases the global stiffness and the slope should increase.

To infer changes in the global stiffness of the RC-beam, it was necessary to create a baseline for the pristine condition. A baseline is an accurate measurement of a process functionality before any change of an input variable occurs. This data allows comparing the effect of a change in the behavior of the phenomenon being evaluated.  The baseline for the pristine beam was built by using the aforementioned load conditions.

\begin{figure}[h!]
  \caption{\csentence{Experimental setup:  RC-beam in INSTRON universal testing machine with 100 kN load cell, in configuration for bending test. 1) Fluke 112 multimeter. In this assembly, it is used to measure the electrical resistance of the self-sensing material. 2) Micron Optics SM130 fiber optic interrogator. This equipment was used to obtain strain data from the FBG. 3) Connection of the tips of the digital multimeter with the electrodes of the self-sensing material. 4) Cementitious matrix sensor electrodes. 5) Load cell support on the beam. 6) FBG sensor positioned so that the compression to which the sensor is exposed is measured when a load is applied to the beam. 7 and 8) Supports for three-point bending measurement.}
      }
      \label{fig6}
      \end{figure}

After building the baseline for the RC-beam, a positive artificial damage (addition of stiffness to the cross section) was induced on it. In this sense, a steel plate was adhered to one of the beam surfaces so as to increase the stiffness of the cross section by $20\%$. The dimensions of the steel plate were 160 $\times$ 160 mm and 10 mm thick. 

The steel plate was bonded to the surface using an epoxy adhesive, for which it was necessary to prepare both surfaces by an abrasive process (to flatten the surface) and cleaning (in order to remove dirt and rust). Afterwards, to keep the steel plate in contact with the RC-beam while the epoxy adhesive cured, "C-clamps" were used.

Same load conditions used for the pristine condition were used for testing the “damaged” RC-beam. Data for the variation of the electrical resistance of the CBM/CNT parallelepiped were acquired whilst load was applied. The experiments for the pristine and damaged states were repeated 10 times each one. The experimental set-up shown in Figure \ref{fig3}b was implemented to measure the variation of the electrical resistance for both states (pristine and damaged).

\section{Results and discussion}

\subsection{Image analysis of the CNT dispersion within the CBM.}

Results from SEM images at a 10 $\mu m$ scale are presented in Figure \ref{fig7}. As expected, the growth of agglomerations or clusters of CNTs within the CBM decrease as the $\%wt$ of CNT increases as can be seen from Figure \ref{fig7}a  to \ref{fig7}c. This result entails as the CNT percentage within the sample increases, the distribution of CNTs within the sample tends to become uniform, since CNT agglomerations generate larger clusters that cover larger areas, producing a uniform lattice within the entire composite \cite{Garcia-Macias2017, Nam2015}.

\begin{figure}[h!]
  \caption{\csentence{SEM images for samples: a) 0.2 $\%wt$ (2700X), b) 0.5 $\%wt$ (2700X) and c) 0.8 $\%wt$ (2000X)}
      }
      \label{fig7}
      \end{figure}


In Figure \ref{fig7}a it can be observed that CNTs are agglomerated in few clusters, also these are randomly scattered on the surface and distant from each other, so it is estimated that the electrical conduction phenomenon occurs mostly by electron hopping \cite{Balberg1984, Garcia-Macias2017} and consequently, this sample is below the percolation threshold. 

Figure \ref{fig7}b shows CNT clusters homogeneously distributed. Additionally, the distance among clusters decreases compared with Figure \ref{fig7}a, and it could be expected that the electrical conductivity phenomenon occurs both by electron hopping and by conductivity networks among CNTs. This behavior indicates that the sample with 0.5 $\%wt$ is above the percolation threshold \cite{Balberg1984, Garcia-Macias2017}.

Furthermore, observing Figure \ref{fig7}c, it can be seen that there is no significant difference or separation between the CNT clusters when the sample has 0.8$\%wt$, so it can be affirmed that this sample is above the percolation threshold. Accordingly, it can be estimated that the qualitative results presented in Figure \ref{fig7} allows the percolation threshold to be estimated between 0.2 $\%wt$ and 0.5 $\%wt$ for this type of composite as was reported by \citet{Souri2017, Garcia-Macias2017, Hoseini2017}. 

None of the Figures \ref{fig7}b and \ref{fig7}c exhibited isolated clusters which are signs of issues caused by agglomeration associated with a poor CNT dispersion within the CBM \cite{Nam2015, Wang2015}. Hence, it can be affirmed that the CNTs dispersion inside the samples with 0.5 $\%wt$ and 0.8 $\%wt$ is homogeneous. 

Finally, the images shown in Figure \ref{fig7} allow to propose two hypotheses regarding the implemented manufacturing method:

\begin{itemize}
    

\item  There is a uniform distribution of CNTs within the cementitious matrix, at least for samples with 0.5  and 0.8$\%wt$, validating the proportions of the constituents selected to make the CBM/CNT material and, in general, the methodology used to disperse the CNTs within the CBM.
\item 	The higher the CNT fraction, the greater the number of clusters which are close to each other, and, therefore, greater the number of CNTs in contact, which entails a greater electrical conductivity.

\end{itemize}


\subsection{Ohmic characterization of the CBM/CNT}
The Ohmic behavior of the material was studied using the I-V curves described in the section \ref{Ohm} and schematized in Figure \ref{fig3}a. From Figure \ref{fig8}a to Figure \ref{fig8}c, it is shown the amperometric response (I-V curves) of the CBM/CNT when an electrical potential is applied. These figures shown a linear relationship between voltage and current, therefore, it is concluded that the CBM/CNT material having 0.2 , 0.5 and 0.8 $\%wt$, exhibit an electrical behavior equivalent to an ideal resistor. From this result, it was determined that any technique or methodology for characterizing the electrical resistance that assumed the material behavior as an Ohmic material can be implemented.  

\begin{figure}[h!]
  \caption{\csentence{Curves I-V using DC for samples with a) 0.2 $\%wt$ of CNTs, b) 0.5 $\%wt$ and c) 0.8 $\%wt$.}
      }
      \label{fig8}
      \end{figure}
  

An approximate measure of the electrical resistance of each specimen obtained using  I-V curves and Ohm's law is presented in Table \ref{tab1}. Nevertheless, the electric resistance measurements of the samples with 0.2  and 0.5 $\%wt$ of CNTs presented a high standard deviation, which implies low reliability, concluding that it is necessary to implement measurement methods of the electrical resistance with higher accuracy. Moreover, it can be seen a change up to four orders of magnitude in the value of electrical resistance when the fraction of CNTs increases from 0.2  to 0.8 $\%wt$, exposing that the percolation threshold is between 0.2 and 0.8 $\%wt$, as also reported by \citet{Garcia-Macias2017}.

\begin{table*}[ht]
\caption{Electrical resistance values and their corresponding standard deviations obtained from the I-V curves characterization.}
\centering
\begin{tabular}{|c|c|c|}
\hline
\textbf{CNTs ($\% wt$)} & \textbf{Average Electrical Resistance ($\Omega$)} & \textbf{Standard deviation ($\Omega$)}\\
\hline
0.2 & 5335505.50 & 250871.25\\
\hline
0.5 & 88245.70 & 3813.08\\
\hline 
0.8 & 693.32 & 14.99\\
\hline 

\end{tabular}
\label{tab1}
\end{table*}

\subsection{Electrical resistance characterization of the CBM/CNT using DC/BDC methodologies}

Results showed in Figure \ref{fig9} were obtained following the experimental setup for electrical characterization. Around this figure, an analysis is presented regarding the relevance of methodologies based on the use of DC and BDC sources against the decrease in data dispersion and accuracy increasing in the electrical resistance characterization of the samples, both against the time.

\begin{figure}[h!]
  \caption{\csentence{Figure 9. Comparison using direct current and BDC for samples of a) 0.2 $\%wt$ of CNTs, b) 0.5 $\%wt$ and c) 0.8 $\%wt$.}
      }
      \label{fig9}
      \end{figure}

It can be seen that the electric polarization phenomena appears when measurements are made using DC methods for 0.2 $\%wt$  and 0.5 $\%wt$. Figure \ref{fig9}a showed a constant increase of the electrical resistance against the time, this phenomena does not tend to stabilize before 40 seconds. Likewise, Figure \ref{fig9}a also exhibit electrical polarization, however, it tends to stabilize after 40 seconds.

The polarization phenomena in the CBM/CNT material with 0.2 $\%wt$ and 0.5 $\%wt$ is due to dielectric nature of the matrix and the low CNT concentration. To explain the polarization phenomena present into CBM/CNT, its necessary understand that the electrical behavior of the  material is determined by two fundamental factors:  the electrical conductivity phenomenon within the material (which it is mainly dominated by electronic conduction), and the capacitance formed between the CNTs and the cementitious matrix \cite{Balberg1984}.

For samples around the percolation threshold (0.2 $\%wt$  and 0.5 $\%wt$), the low CNT concentration allows the CBM occupying spaces among conductive clusters of CNTs and hence, many capacitor-type arrangements are created within the material. As a result, the charge of the formed “capacitors”  causes the electrical polarization phenomena that it is reflected in a constant growth in electrical resistance over time \cite{Dong2016}.

In this context, the CNT concentration must be raised in order to avoid the polarization effect and thus, be able to obtain electrical resistance measurements using DC methods, as is seen in samples with 0.8 $\%wt$ (Figure \ref{fig9}c), where no electric polarization effect is observed.

As was describe in experimental procedure (section \ref{DC_BDC}), BDC methodology was performed in order to avoid polarization phenomena. When analyzing the electrical resistance behavior of the CBM/CNT material characterized by BDC, it is observed that the samples with 0.5 $\%wt$ (Figure \ref{fig9}b) do not exhibit electrical polarization phenomena. In other words, their electrical resistance is invariant throughout time, validating in this way, the methodology  to avoid the polarization proposed by  \citet{Downey2017a, DAlessandro2017}.

Another aspect that is worth analyzing in regard the results presented in Figures \ref{fig9}a, \ref{fig9}b and \ref{fig9}c is the magnitude of the electrical resistance. The measurements obtained with DC are higher than the measurements obtained with BDC. This is due the effect that the electrodes have on the electrical resistance of the material, when DC is used. By contrast, when measurements are made with BDC, and the I-V curve method is used, the contribution to the electrical resistance owing to the electrodes is negligible, as was reported by  \citet{Konsta-Gdoutos2014}.

For samples with 0.2$\%wt$, the characterization based on BDC was not carried out due to the low electrical conductivity. When trying to measure the voltage $V_INT$, the characteristic noise signal of the oscilloscope was higher than the magnitude of the voltage, hence, there was no reliability in the measurements. 

\subsection{Evaluation of the CBM/CNT piezoresistive behavior.}

In this subsection, the relationship between strain and the change in the electrical resistance of the CBM/CNT material when the samples are subjected to a compressing test is established. DC-based methodology  was selected for characterize piezoresistive behavior since BDC equipment required to measure the composite electrical resistance is a scarce resource, and most are custom-made, resulting in an instrumentation price increase and hindering industrial scaling.

Also, piezoresistive characterization was only performed on samples which did not exhibit an electrical polarization effect, since it is impractical to wait until the stabilization time of electrical resistance in samples that exhibited electrical polarization, when an SHM application are procured. In this regard, only characterization of the piezoresistive behavior for the sample with 0.8 $\%wt$ was performed.

The experimental results for the sample with 0.8 $\%wt$ are shown in Figure \ref{fig10}. In this figure, the piezoresistive response of the CBM/CNT is observed as repeatable and reversible since the load do not exceed the CBM/CNT elastic region. In this way, phenomena such as hysteresis or variation of the electrical response will not appear due to repeated load application unless a damage occurs.  

It is also remarkable the proportionality between the electrical resistance and the compression-induced load. The negative slope in the electric response of the CBM/CNT when it is subjected to compressing loads (due to a negative change in the electrical resistance) which is caused by the reduction of the distance among the CNTs when a compressing test is applied to the sample \cite{DAlessandro2017}. This behaviors  make easier the implementation of the CBM/CNT for SHM usages \cite{Lagason2016}.


\begin{figure}[h!]
  \caption{\csentence{ Piezoresistive phenomenon characterization for sample with 0.8 $\%wt$ of CNTs.}
      }
      \label{fig10}
      \end{figure}

Applying Equation \ref{eq1} to all the eight repetitions presented in Figure \ref{fig10}, it was possible to calculate the average gauge factor value ($972.87\epsilon^{-1}$) and its standard deviation  ($17.47 \epsilon^{-1}$). Based on this result, the repeatability of the CBM/CNT material with 0.8 $\%wt$ is demonstrated. Besides, it is observed that the CBM/CNT electric resistance starts to vary from the moment when load is applied, allowing to measure small strains magnitudes.


Observation of Figure \ref{fig10} suggest the ability of the CBM/CNT material to withstand strains up to -100 $\mu \epsilon$  without affecting the structural integrity of the sample, since microcracking or plastic deformation would modify the overall electric resistance and the piezoresistive behavior of the material. This result is in line with the CBM/CNT mechanical characterization carried out following the  ASTM C109/C109M-16a and ACI 318-14 standards, were a compressive strength of 30 MPa and a Young's modulus  of 20 GPa were determined. 

With the previous idea in mind, embedding the CBM/CNT within the RC-beam causes a slight positive stress riser since its stiffness is greater than the concrete commonluy used in the construction industry (i.e. 21 Mpa according to ACI 318 standard). However, it does not represent a threat to the component operation. 

\subsection{ Health monitoring of the reinforced concrete beam using the CBM/CNT material.}

Figure \ref{fig11} shows the change in the electrical resistance measured in the CBM/CNT material, when a load from 0 to 15 kN is applied at a constant displacement speed of 0.5 mm/min. As can be seen, such behavior is nonlinear in the range from 0 to -10 $\mu \epsilon$. This can be associated to the complex phenomena of load transfer through the different interfaces between the RC-beam elements: 1. corrugated steel reinforcement of the RC-beam and the concrete of the RC-beam, 2. concrete of the RC-beam and corrugated steel rods embedded within the parallelepiped (dowels), 3. corrugated steel rods embedded within the parallelepiped (dowels)  and CBM/CNT material of the parallelepiped, and finally, 4. steel plate used as positive damage and concrete of the RC-beam. In particular, the interface between the steel plate and the concrete was constituted by an epoxy adhesive layer, that could contribute greatly to the appearance of the nonlinear phenomena.

\begin{figure}[h!]
  \caption{\csentence{ Behavior of the piezoresistive response of the CBM/CNT against the state of damage and no damage for the RC-beam. Dynamic load from 0 to 15 kN.}
      }
      \label{fig11}
      \end{figure}


From Figure \ref{fig11}, it is possible to observe a change in the slope for damaged state vs. baseline. Such change can be interpreted as a variation in the global beam stiffness, which is promoted by damage occurrence. This result demonstrates the capability to detect global changes in the stiffness of a simple structure such as a beam, by means of strain measurements obtained from a self-sensing. Also ,Figure \ref{fig11}  entails that the CBM/CNT material has the ability to sense small load magnitudes,since the electric resistance experienced a variation associated with small load magnitudes.

To validate the strains obtained through the CBM/CNT parallelepiped embedded into the RC-beam, a FBG sensor was bonded to one of the beam sides. Strains measurements were gathered  whilst electrical resistance was measured by using a DC multimeter. The results obtained are presented in Figure \ref{fig12}.

From Figure \ref{fig12}, it is possible to observe the repetitiveness of the results obtained for the CBM/CNT material. On the basis of the foregoing, the CBM/CNT as the FBG, provide a dynamic response to the load application, wherewith a strain is induced. However, it is observed that there are differences in the strain magnitude as the applied load increases. These differences are mainly due to three factors: 

\begin{enumerate}

 \item  The two sensors are not located at the same place (see numbers 4 and 6 in Figure \ref{fig6}). Therefore, both measurements account for different behaviors at two different locations of the beam,

 \item  The nonlinear load transfer phenomena as explained in section \ref{SHM_beam}, 


\end{enumerate}

\begin{figure}[h!]
  \caption{\csentence{ Behavior of the CBM/CNT and the FBG sensor as function of time. Dynamic load from 0 to 15 kN.}
      }
      \label{fig12}
      \end{figure}

Figure \fig{fig13} shows the results of the strain measured by the FBG for a dynamic load from 0 to 15 kN at a constant displacement speed of 5 mm/min for the damage and undamaged states. Here, two phenomena can be observed: the first one is the change in slope at low load magnitudes (i.e. loads less than 8kN), whereby it matches with the previously described nonlinear behavior for the loads transfer on the RC-beam. The second phenomenon is the global change in slope observed when the structure is damaged, compared to the pristine structure. 

\begin{figure}[h!]
  \caption{\csentence{ Behavior of the FBG vs Load for undamaged and damaged conditions.}
      }
      \label{fig13}
      \end{figure}

In contrast to Figure \ref{fig13}, it is observed that the curves shown in Figure \ref{fig11} do not show a significant slope when the applied load is less than 8kN; this is precisely due to the fact that the sensor were embedded within the structure, measuring the strain through a change in its volume, then, it is expected that the behavior of the acquired data is mostly linear as function of the applied load.


\section*{Conclusions}

In the present work, the electric and self-sensing properties of a CBM/CNT material was evaluated in order to determine the feasibility of implementing as strain and fault detection sensor. In addition, the piezoresistive behavior of CBM/CNT was evaluated both individually and embedded in a RC-beam. The conclusions derived from the present research can be summarized as follow.

\begin{enumerate}
    

\item  The dispersion of CNTs within the CBM can be considered as homogeneous for specimens with concentrations of 0.2 $\%wt$, 0.5 $\%wt$ and 0.8 $\%wt$, when using the manufacturing technique presented in this work.
\item  The electrical behavior of the CBM/CNT material developed can be described as an ideal resistor, consequently, any method of electrical characterization based on Ohm's law can be used to characterize the electrical resistance. In this way, it is concluded that the most suitable characterization method for electrical resistance is provide by DC methodology, since with this method it is possible to obtain dynamic measurements, the standard deviation is smaller and experimental setups are simpler and cheaper compared to methods that use a DC-biphasic type power supply.
\item  Electrical polarization phenomenon occurs only when there are low fractions of CNTs and when characterization techniques of the electrical resistance based on the use of a DC source are used, thus, it is concluded that in order to obtain measurements without electrical polarization, it is necessary to make samples with concentrations of 0.8 $\%wt$ or higher.
The developed CBM/CNT material shows a dynamic self-sensing response, linear and repeatable as function of load application. This makes it attractive for applications in structural health monitoring.
\item  The self-sensing response of the CBM/CNT parallelepiped when it is embedded within a civil structure component such as a beam, is similar to the response exhibited by the cubic samples used for characterizing the piezoresistive effect.  In both of them, the behavior is linear and repeatable, that is why the operation of the CBM/CNT material is validated as a strain sensor.
\item  It was found that by using the CBM/CNT material as a strain sensor, it is possible to infer the damage occurrence in a civil structural component (RC-beam) though studying the change of slope in a load vs. strain curve which can be associated to a global stiffness change promoted by damage occurence. This probes the concept of the first SHM stage (damage detection).
\end{enumerate}


%\nocite{oreg,schn,pond,smith,marg,hunn,advi,koha,mouse}

%%%%%%%%%%%%%%%%%%%%%%%%%%%%%%%%%%%%%%%%%%%%%%
%%                                          %%
%%                                          %%
%%                                          %%
%%%%%%%%%%%%%%%%%%%%%%%%%%%%%%%%%%%%%%%%%%%%%%

\begin{backmatter}

\section*{Competing interests}
  The authors declare that they have no competing interests.

\section*{Author's contributions}

DLCS and JSP  designed the experiments and geometry samples; DLCS and  JAM performed the experiments; DLCS, JSP, JAM and VMT analyzed the data;  and DLCS, JSP, JAM and VMT wrote the paper. All authors read and approved the final manuscript.


\section*{Acknowledgements}

The authors acknowledge support from Professor Dr. Mónica Lucía Álvarez-Láinez, in the Department of Product Design Engineering at Universidad EAFIT.

%%%%%%%%%%%%%%%%%%%%%%%%%%%%%%%%%%%%%%%%%%%%%%%%%%%%%%%%%%%%%
%%                  The Bibliography                       %%
%%                                                         %%
%%  Bmc_mathpys.bst  will be used to                       %%
%%  create a .BBL file for submission.                     %%
%%  After submission of the .TEX file,                     %%
%%  you will be prompted to submit your .BBL file.         %%
%%                                                         %%
%%                                                         %%
%%  Note that the displayed Bibliography will not          %%
%%  necessarily be rendered by Latex exactly as specified  %%
%%  in the online Instructions for Authors.                %%
%%                                                         %%
%%%%%%%%%%%%%%%%%%%%%%%%%%%%%%%%%%%%%%%%%%%%%%%%%%%%%%%%%%%%%

% if your bibliography is in bibtex format, use those commands:
\bibliographystyle{unsrtnat} % Style BST file (bmc-mathphys, vancouver, spbasic).
\bibliography{Art_REF}      % Bibliography file (usually '*.bib' )
% for author-year bibliography (bmc-mathphys or spbasic)
% a) write to bib file (bmc-mathphys only)
% @settings{label, options="nameyear"}
% b) uncomment next line
%\nocite{label}

% or include bibliography directly:
% \begin{thebibliography}
% \bibitem{b1}
% \end{thebibliography}

%%%%%%%%%%%%%%%%%%%%%%%%%%%%%%%%%%%
%%                               %%
%% Figures                       %%
%%                               %%
%% NB: this is for captions and  %%
%% Titles. All graphics must be  %%
%% submitted separately and NOT  %%
%% included in the Tex document  %%
%%                               %%
%%%%%%%%%%%%%%%%%%%%%%%%%%%%%%%%%%%

%%
%% Do not use \listoffigures as most will included as separate files



%%%%%%%%%%%%%%%%%%%%%%%%%%%%%%%%%%%
%%                               %%
%% Tables                        %%
%%                               %%
%%%%%%%%%%%%%%%%%%%%%%%%%%%%%%%%%%%

%% Use of \listoftables is discouraged.
%%


%%%%%%%%%%%%%%%%%%%%%%%%%%%%%%%%%%%
%%                               %%
%% Additional Files              %%
%%                               %%
%%%%%%%%%%%%%%%%%%%%%%%%%%%%%%%%%%%

\section*{Additional Files}
  \subsection*{Additional file 1 --- Sample additional file title}
    Additional file descriptions text (including details of how to
    view the file, if it is in a non-standard format or the file extension).  This might
    refer to a multi-page table or a figure.

  \subsection*{Additional file 2 --- Sample additional file title}
    Additional file descriptions text.


\end{backmatter}
\end{document}
